\documentclass{beamer}
%
% Choose how your presentation looks.
%
% For more themes, color themes and font themes, see:
% http://deic.uab.es/~iblanes/beamer_gallery/index_by_theme.html
%
\mode<presentation>
{
  \usetheme{Madrid}      % or try Darmstadt, Madrid, Warsaw, ...
  \usecolortheme{seahorse} % or try albatross, beaver, crane, ...
  \usefonttheme{serif}  % or try serif, structurebold, ...
  \setbeamertemplate{navigation symbols}{}
  \setbeamertemplate{caption}[numbered]
} 

\usepackage[english]{babel}
\usepackage{kotex}
\usepackage{tikz}
\usepackage{listings}
\usepackage{pgffor}
\input{algorithmbisEnv.tex}

\title[3D 그래픽스 프로그래밍]{그래픽스 강의노트 03 - OpenGL 소개}
\author{강영민}
\institute{동명대학교}
\date{2015년 2학기}

\begin{document}

%%%%%%%%%%%%%%%%%%%%%%%%%%%%%%%%%%%%%%%%%%%%%%%%%%%%%%%%%
\begin{frame}
  \titlepage
\end{frame}

% Uncomment these lines for an automatically generated outline.
%\begin{frame}{Outline}
%  \tableofcontents
%\end{frame}


%%%%%%%%%%%%%%%%%%%%%%%%%%%%%%%%%%%%%%%%%%%%%%%%%%%%%%%%%%
%\begin{frame}[fragile]{깊이 버퍼와 이중 버퍼 사용 예제}
%   \lstset{language=C++,frame=none,escapechar=^}%
%    \foreach \n in {1,26,...,50} {%
%       \only<+>{%
%            \edef\m{\the\numexpr\n+24\relax}%
%            \edef\thesubtitle{{Lines \n--\m\ / 50}}%
%            \expandafter\framesubtitle\thesubtitle
%            \lstinputlisting[firstline=\n,lastline=\m]{./Codes/L03_depthAndDoubleBuffers.tex}%
%       }%
%    }
%\end{frame}
%%%%%%%%%%%%%%%%%%%%%%%%%%%%%%%%%%%%%%%%%%%%%%%%%%%%%%%%%%

%%%%%%%%%%%%%%%%%%%%%%%%%%%%%%%%%%%%%%%%%%%%%%%%%%%%%%%%%
%\begin{frame}[fragile]{간단한 OpenGL 프로그램 테스트}
%\lstset{language=C++,escapechar=^} 
%\begin{lstlisting}
%#include "headers.h"
%
%void myDisplay() {
%   glClear(GL_COLOR_BUFFER_BIT);
%    glFlush();    
%}
%\end{lstlisting}
%\end{frame}
%%%%%%%%%%%%%%%%%%%%%%%%%%%%%%%%%%%%%%%%%%%%%%%%%%%%%%%%%%


%%%%%%%%%%%%%%%%%%%%%%%%%%%%%%%%%%%%%%%%%%%%%%%%%%%%%%%%%
\begin{frame}{OpenGL을 사용하기 위한 준비}


\end{frame}
%%%%%%%%%%%%%%%%%%%%%%%%%%%%%%%%%%%%%%%%%%%%%%%%%%%%%%%%%


\end{document}



