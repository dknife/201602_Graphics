^{\sf [[header 파일 포함하기...]]}^

void init(int argc, char **argv) {
  // ^{\it 윈도우 생성, 버퍼 설정}^
  glutInit(&argc, argv);
  glutInitDisplayMode(GLUT_SINGLE|GLUT_RGBA);
  glutInitWindowPosition(0,0);
  glutInitWindowSize(512,512);
  glutCreateWindow("Dr. Kang's Graphics Lecture");
  glClearColor(1.0, 1.0, 1.0, 1.0);
  // ^{\it 카메라 투영 특성 설정 (glPerspective 사용). 이때는 GL\_PROJECTION 행렬모드여야 한다.}^
  glMatrixMode(GL_PROJECTION);
  glLoadIdentity();
  gluPerspective(60, 1.0, 0.1, 100.0);
}
void drawScene() {
  ^{\sf [[앞서 사용한 코드 \ref{code:OGL_opengl:simpleScene}의 그리기 코드를 여기에 넣는다. (단, glFlush는 여기서 사용하지 않는다)]]}^
}
void drawAxes() {
  glBegin(GL_LINES);
  glColor3f(1.0, 0.0, 0.0);
  glVertex3f(0.0, 0.0, 0.0);   glVertex3f(1.0, 0.0, 0.0);
  glColor3f(0.0, 1.0, 0.0);
  glVertex3f(0.0, 0.0, 0.0);  glVertex3f(0.0, 1.0, 0.0);
  glColor3f(0.0, 0.0, 1.0);
  glVertex3f(0.0, 0.0, 0.0);  glVertex3f(0.0, 0.0, 1.0);
  glEnd();
}
void display() {
  static float t=0.0;
  // ^{\it 카메라의 위치와 방향을 설정한다. 이때는 GL\_MODELVIEW 행렬 모드여야 한다.}^
  glMatrixMode(GL_MODELVIEW);
  glLoadIdentity();
  gluLookAt( 2.0*sin(t), 1, 2.0*cos(t), 0, 0, 0,  0, 1, 0);
  t+=0.001;
  glClear(GL_COLOR_BUFFER_BIT);
  drawScene();		
  glLineWidth(5);
  drawAxes();		
  glLineWidth(1);
  glFlush();
};

int main (int argc, char **argv) {
  init(argc, argv);
  glutDisplayFunc(display);
  glutIdleFunc(display);
  glutMainLoop();
}